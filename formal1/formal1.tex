\documentclass[a4paper,oneside, 11pt]{article}
\usepackage[margin=0.7in]{geometry}
\usepackage[cm-default]{fontspec}
\usepackage{xunicode}
\usepackage{xltxtra}
\usepackage{xgreek}
\usepackage{listings}
\usepackage{hyperref}
\usepackage{parallel,enumitem}
\lstset{basicstyle=\footnotesize\ttfamily,breaklines=true}
\setmainfont[Mapping=tex-text]{CMU Serif}
\usepackage{graphicx}
\usepackage{color}
\usepackage{mathtools, amsmath}
\usepackage{amsfonts}
\usepackage{float}
\usepackage{caption}
\usepackage{titling}
\definecolor{codegreen}{rgb}{0,0.6,0}
\definecolor{codegray}{rgb}{0.5,0.5,0.5}
\definecolor{codepurple}{rgb}{0.58,0,0.82}
\definecolor{backcolour}{rgb}{0.95,0.95,0.92}

\lstdefinestyle{mystyle}{
%    backgroundcolor=\color{backcolour},   
%    commentstyle=\color{codegreen},
%    keywordstyle=\color{magenta},
    numberstyle=\tiny\color{codegray},
%    stringstyle=\color{codepurple},
    basicstyle=\footnotesize,
    breakatwhitespace=false,         
    breaklines=true,                 
    captionpos=b,                    
    keepspaces=true,                 
    numbers=left,                    
    numbersep=5pt,                  
    showspaces=false,                
    showstringspaces=false,
    showtabs=false,                  
    tabsize=2
}
 
\lstset{style=mystyle}
\makeatletter
\def\maxwidth{%
  \ifdim\Gin@nat@width>\linewidth
    \linewidth
  \else
    \Gin@nat@width
  \fi
}
\makeatother

\makeatletter
\newcommand\xlongleftrightarrow[2][]{%
\ext@arrow 0059{\longleftrightarrowfill@}{#1}{#2}%
}
\def\longleftrightarrowfill@{%
\arrowfill@ ← \relbar → }
\makeatother




\pretitle{%
		\begin{center}
		\LARGE
		\includegraphics[height=7cm]{figures/pyrforos.png}\\[\bigskipamount]
}
\posttitle{\end{center}}
\title{\textbf{}}
\author{ Ιωάννης Δάρας (\texttt{03115018, el15018@central.ntua.gr, daras.giannhs@gmail.com}) \\
}
\date{}

\newtheorem{theorem}{Θεώρημα}
\begin{document}
\maketitle
\noindent\makebox[\linewidth]{\rule{\paperwidth}{0.4pt}}



\section*{Άσκηση 1}
\subsection{}
$$
E[X_t] = E[Asin(\omega t + \Theta)] = E[A] \cdot E[sin(\omega t + \Theta)]
$$
Στην παραπάνω παράσταση χρησιμοποιήσαμε το εξής θεώρημα:
\begin{theorem}
	Αν δύο τυχαίες μεταβλητές X, Y είναι ανεξάρτητες, τότε $E[XY] = E[X] E[Y]$
\end{theorem}
$$
E[X_t] = \frac{1}{\lambda}E[sin(\omega t + \Theta)] = E[sin(\omega t + \Theta)]
$$
$$
E[X_t] = \int_{-\infty}^{\infty} sin(\omega t + \theta) f_{\Theta}(\theta) \textrm{d}\theta = \int_{0}^{2\pi} sin(\omega t + \theta) \frac{1}{2\pi}\textrm{d}\theta = 0$$
\subsection{}
$$
E[X_t \cdot X_s] = E[A^2 \cdot sin(\omega t + \Theta) \cdot sin(\omega s + \Theta)] 
= E[A^2] \cdot E[sin(\omega t + \Theta) \cdot sin(\omega s + \Theta)]$$
Στην παραπάνω παράσταση χρησιμοποιήσαμε το Θεώρημα 1 και το Θεώρημα 2 που παραθέτουμε παρακάτω.
\begin{theorem}
	Αν δύο τυχαίες μεταβλητές $X, Y$ είναι ανεξάρτητες τότε και $g(X), f(Y)$ ανεξάρτητες όπου $g, f$ οποιεσδήποτε συναρτήσεις.
\end{theorem}

Τα θεωρήματα 1, 2 είναι γνωστά από το μάθημα των Πιθανοτήτων και συνεπώς η απόδειξη τους παραλείπεται.
Έχουμε:
$$
V(A) = E[A^2] - E[A]^2
$$
Όμως: $A \sim Exp(\lambda)$ για την οποία γνωρίζουμε ότι: $$V[A]=\frac{1}{\lambda^2}$$
$$
E[A] = \frac{1}{\lambda}
$$
Άρα:
$$
E[A^2] = \frac{2}{\lambda^2}
$$
Έτσι, έχουμε:
$$
E[X_t\cdot X_s] = E[sin(\omega t + \Theta) \cdot sin(\omega s + \Theta)] \cdot \frac{2}{\lambda^2} = \frac{1}{\lambda^2} E[cos(\omega (t-s)) - cos(\omega (t+s) + 2\Theta)]
$$
$$
E[X_t\cdot X_s] = cos(\omega (t-s)) - E[cos(\omega (t+s) + 2\Theta)]
$$
Στην παραπάνω παράσταση ο δεύτερος όρος μηδενίζεται με πανομοιότυπο με εκείνον που ακολουθήσαμε στο ερώτημα 1.1. Έτσι, προκύπτει τελικά ότι:
$$
E[X_t \cdot X_s] = cos\big( \omega (t-s)\big)
$$



\section*{Άσκηση 5}
Η άσκηση αυτή αναλύθηκε στην 1η εργαστηριακή άσκηση. Μεταφέρουμε τον πίνακα πιθανοτήτων μετάβασης από την άσκηση αυτή:
$$P = \begin{bmatrix}0 & 0.5 & 0.5 & 0 & 0  \\ 0.5 & 0 & 0.5 & 0 & 0 \\ 0.25 & 0.25 & 0 & 0.25 & 0.25 \\ 0 & 0 & 1 & 0 & 0\\ 0 & 0 & 1 & 0 & 0\end{bmatrix}$$


\section*{Άσκηση 7}
\subsection{}
Η κάθε κατάσταση κωδικοποιεί τον αριθμό των συνεχόμενων 6αριών που έχουμε φέρει. Έτσι, από μια κατάσταση με πιθανότητα $\frac{1}{6}$ πάμε στην επόμενη της (αν φέρουμε 6) ή με πιθανότητα $\frac{5}{6}$ (αν φέρουμε οτιδήποτε άλλο) γυρίζουμε στην κατάσταση 0. Η κατάσταση 5 μένει με πιθανότητα 1 στον εαυτό της, αφού το πείραμα έχει τελειώσει. Ο πίνακας πιθανοτήτων μετάβασης είναι:
$$
\mathcal{P} = \begin{bmatrix}
5/6 & 1/6 & 0 & 0 & 0 & 0\\
5/6 & 0 & 1/6 & 0 & 0 & 0 \\
5/6 & 0 & 0 & 1/6 & 0 & 0 \\
5/6 & 0 & 0 & 0 & 1/6 & 0 \\
5/6 & 0 & 0 & 0 & 0 & 1/6 \\
0 & 0 & 0 & 0 & 0 & 1
\end{bmatrix}
$$

\subsection{}
Σε αυτό το πείραμα, οι καταστάσεις $\mathbb X$ μοντελοποιούν τα ακόλουθα:
\begin{itemize}
	\item 0: κανένα στοιχείο της ζητούμενης ακολουθίας
	\item 1: 6
	\item 2: 65
	\item 3: 656
	\item 4: 6565
	\item 5: 65656
\end{itemize}


$$\mathcal{P} = \begin{bmatrix}
5/6 & 1/6 & 0 & 0 & 0 & 0\\
4/6 & 1/6 & 1/6 & 0 & 0 & 0 \\
5/6 & 0 & 0 & 1/6 & 0 & 0 \\
4/6 & 1/6 & 0 & 0 & 1/6 & 0 \\
5/6 & 0 & 0 & 0 & 0 & 1/6 \\
0 & 0 & 0 & 0 & 0 & 1
\end{bmatrix}
$$


\section*{Άσκηση 12}
Στο πείραμα αυτό κάθε κατάσταση του $\mathbb X$ μοντελοποιεί τον αριθμό των σωματιδίων που ανήκουν στη διαμέριση Α. Αφού σε κάθε βήμα η διαμέριση μπορεί να έχει είτε ένα σωματίδιο λιγότερο είτε ένα σωματίδιο περισσότερο, οι πιθανότητες μετάβασης σε μη γειτονικές καταστάσεις είναι 0. Αντίθετα, για τις γειτονικές καταστάσεις η πιθανότητα είναι $\frac{1}{2}$. Φυσικά, για την πρώτη και την τελευταία κατάσταση (0 σωματίδια ή N σωματίδια αντίστοιχα) με πιθανότητα $\frac{1}{2}$ μεταβαίνεις στη γειτονική τους κατάσταση και με $\frac{1}{2}$ μένεις στην ίδια. \bigbreak

Συνοπτικά:
$$
p(i,j) = \begin{cases}
1/2 & |j - i| = 1 \wedge i > 0 \wedge i < N \\
1/2 & j = i = 0 \wedge i = j = N \\
0 & \text{αλλού} 
\end{cases}
$$





\section*{Βιβλιογραφία}
[1] Λουλάκης, Μ., 2015. Στοχαστικές Διαδικασίες. [ηλεκτρ. βιβλ.] Αθήνα:Σύνδεσμος Ελληνικών Ακαδημαϊκών Βιβλιοθηκών

\end{document}
