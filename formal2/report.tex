\documentclass[a4paper,oneside, 11pt]{article}
\usepackage[margin=0.7in]{geometry}
\usepackage[cm-default]{fontspec}
\usepackage{xunicode}
\usepackage{xltxtra}
\usepackage{xgreek}
\usepackage{listings}
\usepackage{hyperref}
\usepackage{parallel,enumitem}
\lstset{basicstyle=\footnotesize\ttfamily,breaklines=true}
\setmainfont[Mapping=tex-text]{CMU Serif}
\usepackage{graphicx}
\usepackage{color}
\usepackage{mathtools, amsmath, amsfonts}
\usepackage{float}
\usepackage{caption}
\usepackage{titling}
\definecolor{codegreen}{rgb}{0,0.6,0}
\definecolor{codegray}{rgb}{0.5,0.5,0.5}
\definecolor{codepurple}{rgb}{0.58,0,0.82}
\definecolor{backcolour}{rgb}{0.95,0.95,0.92}

\lstdefinestyle{mystyle}{
%    backgroundcolor=\color{backcolour},   
%    commentstyle=\color{codegreen},
%    keywordstyle=\color{magenta},
    numberstyle=\tiny\color{codegray},
%    stringstyle=\color{codepurple},
    basicstyle=\footnotesize,
    breakatwhitespace=false,         
    breaklines=true,                 
    captionpos=b,                    
    keepspaces=true,                 
    numbers=left,                    
    numbersep=5pt,                  
    showspaces=false,                
    showstringspaces=false,
    showtabs=false,                  
    tabsize=2
}
 
\lstset{style=mystyle}
\makeatletter
\def\maxwidth{%
  \ifdim\Gin@nat@width>\linewidth
    \linewidth
  \else
    \Gin@nat@width
  \fi
}
\makeatother

\makeatletter
\newcommand\xlongleftrightarrow[2][]{%
\ext@arrow 0059{\longleftrightarrowfill@}{#1}{#2}%
}
\def\longleftrightarrowfill@{%
\arrowfill@ ← \relbar → }
\makeatother




\pretitle{%
		\begin{center}
		\LARGE
		\includegraphics[height=7cm]{figures/pyrforos.png}\\[\bigskipamount]
}
\posttitle{\end{center}}
\title{\textbf{Στοχαστικές Διαδικασίες \\ 2η Σειρά Γραπτών Ασκήσεων}}
\author{ Ιωάννης Δάρας (\texttt{03115018, el15018@central.ntua.gr, daras.giannhs@gmail.com}) \\
}
\date{}

\newtheorem{theorem}{Theorem}
\begin{document}
\maketitle
\noindent\makebox[\linewidth]{\rule{\paperwidth}{0.4pt}}

\section*{Άσκηση 20}
Η ζητούμενη πιθανότητα είναι η $\pi_n(1)$.


Γνωρίζουμε ότι:
$$
\pi_n =  \pi_0 P^n
$$
Στο πρόβλημα μας, ξεκινάμε από την κατάσταση $X_0=1$, συνεπώς:
$$
\pi_0 = \begin{bmatrix}
1 & 0 & 0
\end{bmatrix}
$$

Προκειμένου να υπολογίσουμε τον πίνακα $P^n$ πρέπει να διαγωνοποιήσουμε τον πίνακα P, δηλαδή να τον γράψουμε στην μορφή:
$$
P = S \cdot \Lambda \cdot S^{-1}
$$ 

 Προκειμένου να τον διαγωνοποιήσουμε, πρέπει πρώτα να βρούμε τις ιδιοτιμές του.
Έχουμε:
$$
P = \begin{bmatrix}
0 & 1/2 & 1/2 \\
1/3 & 1/3 & 1/3 \\
p & 2/3 - p & 1/3
\end{bmatrix}
$$

$$
|P - \lambda I| = 0 \iff 
$$


$$
\begin{vmatrix}
-\lambda & 1/2 & 1/2 \\ 
1/3 & 1/3 - \lambda & 1/3 \\
p & 2/3 - p & 1/3 - \lambda 
\end{vmatrix} = 0 \iff 
$$
$$
-\lambda \begin{vmatrix} 1/3 - \lambda &  1/3 \\ 2/3 - p & 1/3 - \lambda \end{vmatrix} - 1/2 \begin{vmatrix} 1/3 & 1/3 \\ p & 1/3 -\lambda \end{vmatrix} + 1/2 \begin{vmatrix}
1/3 & 1/3 -\lambda \\ p & 2/3 -p \end{vmatrix} = 0 \iff 
$$
$$
-18\lambda ^ 3 + 12\lambda^2 + 3\lambda p - 3p + 5\lambda + 1 =0
$$
Παρατηρούμε ότι το 1 είναι ρίζα της παραπάνω εξίσωσης.
Άρα γράφεται ισοδύναμα:
$$
(\lambda - 1) \cdot (-18\lambda^2 - 6\lambda -1 + 3p) = 0 \qquad (1)
$$
\subsection*{(α)}
Για p = 0 η (1) γράφεται:
$$
(\lambda - 1) \cdot (-18\lambda^2 - 6\lambda -1 ) = 0
$$
Από την παραπάνω εξίσωση προκύπτουν οι ιδιοτιμές:
$$
\lambda_1 = -\frac{1}{6} - \frac{1}{6}j, \quad \lambda_2 = -\frac{1}{6} + \frac{1}{6}j, \quad \lambda_3 = 1
$$
Ο πίνακας ιδιοτιμών είναι λοιπών:
$$
\Lambda = \begin{bmatrix}
-1/6 -1/6j & 0 & 0 \\
0 & -1/6 +1/6j & 0 \\
0 & 0 & 1 
\end{bmatrix}
$$

Για να βρούμε τα ιδιοδιανύσματα λύνουμε την εξίσωση:
$$
(P - \lambda_i I)\cdot v_i = \vec 0
$$
για τα διάφορα $\lambda_i$ που βρήκαμε. \bigbreak 

Έτσι, προκύπτουν τα ιδιοδιανύσματα:
$$
v_1 = \begin{bmatrix}
0.507j & -0.507j & 0.577
\end{bmatrix}, \quad
v_2 = \begin{bmatrix}
	0.507 - 0.17j & 0.507 + 0.17j & 0.577
\end{bmatrix}, \quad 
v_3 = \begin{bmatrix}
	-0.676 & -0.676 & 0.577
\end{bmatrix}
$$
Ο πίνακας S γράφεται:
$$
S = \begin{bmatrix}
0.507j & -0.507j & 0.577 \\
0.507 - 0.17j & 0.507 + 0.17j & 0.577 \\ 
-0.676 & -0.676 & 0.577
\end{bmatrix}
$$
Ο αντίστροφος πίνακας $S^{-1}$ μπορεί να βρεθεί με τη μέθοδο των υποοριζουσών και είναι:
$$
S^{-1} = \begin{bmatrix}
0.118 - 0.828j & 0.354 + 0.473j & -0.473 + 0.355j \\
0.118 + 0.828j & 0.354 - 0.473j & -0.473 - 0.355j \\
0.277 & 0.831 & 0.623
\end{bmatrix}
$$
Έτσι, τελικά έχουμε:
$$
P = S \cdot \Lambda \cdot S^{-1} = 
$$
$$
= \footnotesize{\begin{bmatrix}
0.507j & -0.507j & 0.577 \\
0.507 - 0.17j & 0.507 + 0.17j & 0.577 \\ 
-0.676 & -0.676 & 0.577
\end{bmatrix} \cdot \begin{bmatrix}
-1/6 -1/6j & 0 & 0 \\
0 & -1/6 +1/6j & 0 \\
0 & 0 & 1 
\end{bmatrix} \cdot \begin{bmatrix}
0.118 - 0.828j & 0.354 + 0.473j & -0.473 + 0.355j \\
0.118 + 0.828j & 0.354 - 0.473j & -0.473 - 0.355j \\
0.277 & 0.831 & 0.623
\end{bmatrix}}
$$


Όμως:
$$
P_n = S \cdot \Lambda^n \cdot S^{-1}
$$
που πλέον ο διαγώνιος πίνακας υψώνεται εύκολα στη δύναμη n αφού για να γίνει αυτό αρκεί να υψωθούν τα στοιχεία της διαγωνίου στη δύναμη αυτή.




\subsection*{(β)}
Για p=1/6 από την (1) παίρνουμε τις ακόλουθες ιδιοτιμές:
$$
\lambda_1 = \lambda_2 = -\frac{1}{6}  \textrm{ (διπλή)}, \quad \lambda_3 = 1
$$

Για τα ιδιοδιανύσματα $\lambda_1, \lambda_3$ αρκεί να λύσω την εξίσωση:
$$
(P - \lambda_i I)\cdot v_i = \vec 0
$$

Το $v_2$ είναι γενικευμένο ιδιοδιάνυσμα καθώς προκύπτει από διπλή ιδιοτιμή. Έτσι, λύνουμε την εξίσωση:
$$
(P - \lambda_2 I)\cdot v_2 = v_1 
$$
Κατά τα άλλα δουλεύουμε ακριβώς όπως παραπάνω.
\subsection*{(γ)}
Για p=2/3 από την (1) παίρνουμε τις ακόλουθες ιδιοτιμές:
$$
\lambda_1 = 0, \quad \lambda_2 = -1/6 \cdot (1 - \sqrt3), \lambda_3 = -1/6 \cdot (1 + \sqrt3)
$$

Αφου όλες οι ιδιοτιμές έχουν πολλαπλότητα 1, τα ιδιοδιανύσματα βρίσκονται ακριβώς όπως έχουμε ήδη δει, λύνοντας δηλαδή την εξίσωση:
$$
(P - \lambda_i I)\cdot v_i = \vec 0
$$


\subsection*{(δ)}
Ο υπολογισμός του $\lim_{n\to \infty}P^n$ γίνεται αποδοτικά μέσω της διαγωνοποίησης, δηλαδή:

$$ \lim_{n\to \infty}P^n = S \left( \lim_{n \to \infty}\Lambda ^ n \right) S^{-1} $$


\section*{Άσκηση 21}
\subsection{(α)}
Έχουμε:
$$
P = \begin{bmatrix}
0 & 1/6 & 1/12 & 3/4 \\
1/2 & 0 & 1/4 & 1/4 \\
0 & 1/2 & 0 & 1/2 \\
1/2 & 1/3 & 1/6 & 0
\end{bmatrix}
$$


Θέλουμε να δείξουμε για μεγάλο n υπάρχει ανεξαρτησία της $\pi_n$ ως προς την αρχική κατανομή. Για να το κάνουμε αυτό, πρέπει να υπολογίσουμε τον πίνακα $P^n$, που σημαίνει ότι αρχικά πρέπει να τον διαγωνοποιήσουμε. \bigbreak 

Αρχικά, βρίσκουμε τις ιδιοτιμές του πίνακα λύνοντας την εξίσωση:
$$
|P - \lambda I| = 0
$$ 
Από την παραπάνω εξίσωση παίρνουμε:
$$
\lambda_1 = 1, \quad \lambda_2 = \lambda_3 = -0.5 \textrm{ (διπλή)}, \quad \lambda_4 = 0 
$$
Για να βρούμε τα ιδιοδιανύσματα $v_1, v_2, v_4$ χρησιμοποιούμε την εξίσωση:
$$
(P - \lambda_i I)\cdot v_i = \vec 0
$$
Επειδή το ιδιοδιάνυσμα $v_3$ είναι το δεύτερο ιδιοδιάνυσμα διπλής ιδιοτιμής, είναι γενικευμένο ιδιοδιάνυσμα και προκύπτει από την εξίσωση:
$$
(P - \lambda_3 I)\cdot v_3 = v_2
$$


Από τα παραπάνω έχουμε ότι ο πίνακας των ιδιοτιμών είναι ένας πίνακας Jordan (λόγω της διπλής ιδιοτιμής) και έχει τις τιμές που φαίνονται παρακάτω:
$$
J = \begin{bmatrix}
1 & 0 & 0 & 0 \\
0 & -0.5 & 1 & 0 \\
0 & 0 & -0.5 & 0 \\
0 & 0 & 0 & 0 \\
\end{bmatrix}
$$

Για έναν jordan πίνακα γνωρίζουμε ότι όταν τον υψώνουμε στη δύναμη n τότε το m-οστό στοιχείο μετρώντας από μια ιδιοτιμή $\lambda_i$ προς τα δεξιά είναι:
$$
\binom{n}{m} \lambda_i^{n-m}
$$
Έτσι, έχουμε:
$$
J^n = \begin{bmatrix}
1 & 0 & 0 & 0 \\
0 & (-0.5)^n & n \cdot (-0.5)^{n-1} & 0 \\
0 & 0 & (-0.5)^n & 0 \\
0 & 0 & 0 & 0 \\
\end{bmatrix} = \begin{bmatrix}
1 & 0 & 0 & 0 \\
0 & (-0.5)^n & n \cdot (2)^{1-n} (-1)^{n+1} & 0 \\
0 & 0 & (-0.5)^n & 0 \\
0 & 0 & 0 & 0 \\
\end{bmatrix}
$$


Όμως:
$$
P^n = S \cdot J^n \cdot S^{-1} = S \cdot \begin{bmatrix}
1 & 0 & 0 & 0 \\
0 & (-0.5)^n & n \cdot (2)^{1-n} (-1)^{n+1} & 0 \\
0 & 0 & (-0.5)^n & 0 \\
0 & 0 & 0 & 0 \\
\end{bmatrix} \cdot S^{-1}
$$


Έτσι, προκύπτει ότι:
$$
\lim_{n\to \infty}P^n = S \cdot \begin{bmatrix}
1 & 0 & 0 & 0 \\
0 & 0 & 0 & 0\\
0 & 0 & 0 & 0 \\
0 & 0 & 0 & 0
\end{bmatrix}\cdot S^{-1}
$$


Από τη μορφή του πίνακα εύκολα προκύπτει ότι τα διανύσματα $S, S^{-1}$ θα προσθέσουν μετατοπίσεις που θα είναι ανεξάρτητες από την αρχική θέση $\pi_0$, της οποίας η επίδραση θα σβήσει λόγω του πίνακα μηδενικής ισχύος $J^n$.

Έτσι, έχουμε:
$$
\pi_n \to \pi_*
$$





\section*{Άσκηση 24}
$$
P = \begin{bmatrix}
1/2 & 1/2 & 0 & 0 & 0 & 0 & 0 & 0 \\
1/4 & 3/4 & 0 & 0 & 0 & 0 & 0 & 0 \\
1/4 & 1/4 & 0 & 1/8 & 3/8 & 0 & 0 & 0 \\
0 & 0 & 1/4 & 0 & 3/4 & 0 & 0 & 0 \\
0 & 0 & 1/5 & 1/5 & 1/5 & 1/5 & 1/5 & 0 \\
0 & 0 & 0 & 0 & 0 & 0 & 1/2 & 1/2 \\
0 & 0 & 0 & 0 & 0 & 1/2 & 0 & 1/2 \\
0 & 0 & 0 & 0 & 0 & 1/2 & 1/2 & 0 
\end{bmatrix}
$$


\subsection*{Κλάσεις επικοινωνίας}
Οι κλάσεις επικοινωνίας της στοχαστικής αυτής ανέλιξης είναι:
$$
\mathcal C_1 = \{1, 2\}, \quad \mathcal C_2 = \{3,4,5\}, \quad \mathcal C_3 = \{6,7,8\}
$$

Η κλάση $\mathcal C_{1}$ είναι κλειστή καθώς όλες οι πιθανότητες μετάβασης προς κάποιο στοιχείο εκτός της κλάσης είναι 0. Το ίδιο ισχύει και για την κλάση $\mathcal C_3$. Η κλάση $\mathcal C_2$ είναι ανοικτή καθώς η κατάσταση 3 έχει θετική πιθανότητα μετάβασης προς την κατάσταση 1 και η κατάσταση 1 ανήκει σε κλειστή κλάση. 

\subsection*{Επαναληπτικότητα}
Αφού οι καταστάσεις 1,2,6,7,8 ανήκουν σε κλειστές κλάσεις επικοινωνίας, από το θεώρημα 6, είναι επαναληπτικές καταστάσεις. 

Οι καταστάσεις 3,4,5 ανήκουν σε ανοικτή κλάση επικοινωνίας, συνεπώς, από το θεώρημα 5, είναι παροδικές.


\section*{Άσκηση 27}
$$
T = \mathrm{inf}(k \geq 5: X_k = X_2)
$$
Έχουμε:
$$
\{ T = n \} = \begin{cases}
0, \qquad n<5 \\
X_5 = X_2, \qquad n = 5 \\
(X_5 \neq X_2) \wedge ... \wedge (X_{n-1} \neq X_2) \wedge (X_n = X_2)
\end{cases}
$$
\bigbreak 

Οι πρώτοι δύο κλάδοι ανήκουν στην $\mathcal F_n$ με προφανή τρόπο. Ο τρίτος κλάδος ανήκει στην $\mathcal F_n$ ως τομή πραγμάτων που ανήκουν στην $F_n$ σύμφωνα με την πρόταση 4 του βιβλίου. \bigbreak 

Άρα:
$$
\{ T = n \} \in \mathcal F_n
$$
άρα ο χρόνος T είναι χρόνος διακοπής.

\section*{Άσκηση 29}
Έστω $\{X_n\}_{n \in \mathbb N_o}$ για την οποία το στοιχείο i,j του πίνακα πιθανοτήτων μετάβασης της δίνεται από τη σχέση:
$$
p[i,j] =  \begin{cases}
0.5, \qquad j=i+1 \\ 0.5, \qquad j=i+2 \\ 
0, \qquad \textrm{αλλιώς}
\end{cases}
$$
Αυτή η στοχαστική διαδικασία μοντελοποιεί το πείραμα του να "περπατάμε" στους φυσικούς αριθμούς ρίχνοντας κάθε φορά ένα τίμιο κέρμα για το αν θα πάμε στον επόμενο ή στο μεθεπόμενο αριθμό. \bigbreak 

Σε αυτή τη στοχαστική διαδικασία κάθε αριθμός είναι μια κλάση επικοινωνίας και προφανώς η στοχαστική διαδικασία έχει μόνο ανοικτές κλάσεις επικοινωνίας αφού δεν μπορούμε να ξαναγυρίσουμε σε έναν αριθμό.


\section*{Άσκηση 32}
Έστω ότι:
$$
P_1[T_0 < \infty] =  a
$$
Η πιθανότητα αυτή εκφράζει τη βεβαιότητα μας για το αν θα φτάσουμε στο 0 σε πεπερασμένο χρόνο ξεκινώντας από το 1.
Όμως αντίστοιχα, η πιθανότητα να φτάσουμε στο 1 σε πεπερασμένο χρόνο ξεκινώντας από το 2 είναι η ίδια, από την ισχυρή μαρκοβιανή ιδιότητα, δηλαδή:
$$
P_2[T_1 < \infty] = a
$$
Από κανόνα γινομένου όμως έχουμε ότι:
$$
P_x[T_o < \infty] = P_{x-1}[T_x < \infty] \cdot P_{x-2}[T_{x-1} < \infty] \cdot ... \cdot P_1[T_o<\infty] = a \cdot a \cdot ... \cdot a = a^x
$$
δηλαδή δείξαμε το ζητούμενο.
\end{document}
